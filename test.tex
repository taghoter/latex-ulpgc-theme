\documentclass[a4paper]{beamer}
\usetheme[spanish,helvet]{ulpgc}

\author{Pepe P\'erez}
\title{Charlilla\ldots}
\date{\today}

\begin{document}

\begin{frame}

\titlepage
\end{frame}

\begin{frame}
	\frametitle{\'Indice}
	\tableofcontents
\end{frame}

\section{foo1}
\subsection{foo11}

\begin{frame}{Lorem ipsum}
foo bar
\end{frame}

\begin{frame}{Lorem ipsum}
\begin{block}{Un bloque de muestra}
$$\int_{f}^3 f(x) dx$$
\begin{itemize}
  \item Elemento de ejemplo
  \pause
  \item Otro m\'as
  \footnote{\tiny K.~Eilbeck \emph{et al.} The Sequence Ontology: a tool for 
  the unification of genome annotations. \emph{Genome Biol.}, 6:R44 (2005)}
\end{itemize}
\end{block}
\end{frame}

\section{foo2}
\begin{frame}[plain]{Escucha}
\vfill
\begin{center}\begin{Large}Esta es una transparencia no
vac\'{\i}a del todo.\end{Large}\vfill
\end{center}
\vfill
\end{frame}

\begin{frame}
  \vfill
  \begin{center}\begin{Large}Y \'esta es una transparencia que tampoco est\'a
completamente vac\'{\i}a, pero tiene el \emph{matiz} de no 
tener t\'{\i}tulo.\end{Large}\vfill
  \end{center}
  \vfill
\end{frame}

\section{foo3}
\begin{frame}[<+->]{Escucha}
\framesubtitle{Prueba de escucha 2}
\begin{enumerate}
\item foo
\item foo
\item foo
\begin{enumerate}
\item foo
\item foo
\end{enumerate}
\item foo
\end{enumerate}
\end{frame}

\section{foo5}
\begin{frame}[allowframebreaks,allowdisplaybreaks]{Teoremas, lemas y 
demostraciones}
\begin{proof}
        Demostraci\'on
\end{proof}

\begin{definition}
        Definici\'on
\end{definition}

\begin{lemma}[Enunciado]
        Lema
\end{lemma}

\begin{theorem}[Sentencia]
        Teorema
\end{theorem}
\end{frame}

\section{foo6}
\begin{frame}
\vfill
\begin{center}\begin{Huge}Muchas gracias \\[10pt]
por su atenci\'on.\end{Huge}\vfill
\end{center}
\vfill
\end{frame}

\end{document}
